\documentclass[twoside,10pt]{article}
\usepackage{xparse}
\usepackage[margin=0.9in]{geometry}
\usepackage{tcolorbox}
\setlength\parindent{0pt}
\definecolor{bubblegum}{rgb}{0.99, 0.76, 0.8}
\usepackage[normalem]{ulem}
\usepackage{xcolor,colortbl}
\usepackage{framed}
\usepackage{caption}
\usepackage{subcaption}
\captionsetup{font={small}}
\usepackage{amsmath}
\usepackage{amssymb}
\usepackage{graphicx}
\usepackage{listings}
\usepackage{lipsum}
\usepackage{courier}
\usepackage{xcolor}

\definecolor{backcolour}{rgb}{0.95,0.95,0.92}

\lstdefinestyle{mystyle}{
    backgroundcolor=\color{backcolour},
    basicstyle=\ttfamily\normalsize,
    breakatwhitespace=false,         
    breaklines=true,                 
    captionpos=b
}
\lstset{style=mystyle}


\begin{document}

\begin{center}
    {\Large \bf CS 682 – Artificial Intelligence}

    \vspace{.5cm}

    {\Large \bf Project 2 - Spam Filter}

    \vspace{0.5cm}
    {\large \bf Monikrishna Roy}
    \vspace{0.5cm}

    {\large \today}

\end{center}

\section*{Purpose}\label{purpose}

The purpose of this assignment is to provide a multi-executable
programming experience with a discriminatory algorithm that can make
decisions using Artificial Intelligence. You should be able to utilize
probability in order to classify using a Naive Bayes Classifier.

\section*{Task}\label{task}

Please write a program to parse an existing dataset on real-world SMS
messages. (note: since these data come from real-world interactions,
these messages may use language which I would never use in class and
that violates professional conversational norms. If that is likely to
trigger a negative reaction, please do not read the messages themselves.
However, I feel that it is important to work with real-world data
wherever possible).\\

You will need to write two programs:

\begin{enumerate}
    \item

          \textit{training -i \textless{}spam.csv file\textgreater{} -os \textless{}output spam probability file\textgreater{} -oh \textless{}output ham probability file\textgreater{}}

          Trains dataset from .csv file and save to new file. Each line of the
          .csv file has at least two fields, separated by a comma:

          \begin{verbatim}
    1. <spam|ham> ham if it is a legitimate SMS, spam if not
    2. "..." the SMS message
\end{verbatim}

          You will need to output two probability files (one for ham, one for
          spam):

          \begin{verbatim}
<count of the total number of words (n)>
m lines, one for each word <word> <number of word occurrences>
\end{verbatim}
    \item
          \textit{classify -i \textless{}testing dataset .csv file\textgreater{} -is \textless{}spam probability file\textgreater{} -ih \textless{}ham probability file\textgreater{} -o \textless{}classification output filename\textgreater{}}

          Classifies new data from training file and testing .csv file (same
          format as above, specified on the command-line)

          You will need to output one classification file:

          \begin{verbatim}
m lines, one for each SMS in the testing dataset
(in the same order as the testing set is in <spam/ham>)
(the classification of the SMS)
\end{verbatim}
\end{enumerate}

\section*{Graduate Student Extra
  Assignment}\label{graduate-student-extra-assignment}

Please also write a program to add new data (in a .csv file) to the
existing training database. \\

\textit{addtotraining -is \textless{}input spam probability file\textgreater{} -ih \textless{}input ham probability file\textgreater{} -s "\textless{}string\textgreater{}"}

\section*{Solution/Algorithms}\label{solutionalgorithms}

Naive Bayes Classifier is used to calculate the probability of the
spam/hum messages.

\subsection*{Naive Bayes Classifier}\label{naive-bayes-classifier}

A Naive Bayes classifier is a probabilistic machine learning model that’s used for the classification task. The crux of the classifier is based on the Bayes theorem.\\

The multinomial Naive Bayes method is used here. This is mainly for the document classification problem, i.e., whether a document belongs to the category of sports, politics, technology, etc. The features/predictors used by the classifier are the frequency of the words present in the document.\\

The following formula is used to calculate the probability of a message being spam or not:
\paragraph{Equation}\label{equation}

\begin{verbatim}
    P(w|c) = (C(w, c) + 1) / (C(c) + |V|)

    where,
    P(w|c) = the probability of a word belongs to given class c (ham/spam)
    C(w, c) =  the frequency of the word in the given class c
    C(c) = total number of words in the given class
    |V| = number of the unique vocabulary in the dataset.
\end{verbatim}

\section*{Challenges}\label{challenges}

There are a few challenges for this project. To get good results, data
processing and feature extraction are most important here. In this
project, only text cleaning is done as data processing, but feature
extraction has not been applied.\\

The following steps are done for text cleaning:

\begin{itemize}
    \itemsep1pt\parskip0pt\parsep0pt
    \item convert all letters to lowercase
    \item clean punctuation
    \item clean numbers
    \item remove multiple spaces
    \item remove non-ascii characters
    \item remove not alphabetic characters
    \item remove single characters
    \item remove hyperlinks
\end{itemize}

\section*{Pre-requisites and Environment
  Settings}\label{pre-requisites-and-environment-settings}

\begin{itemize}
    \itemsep1pt\parskip0pt\parsep0pt
    \item Python \textgreater{}= 3.8
    \item Pandas
\end{itemize}

\newpage

\section*{Run Command / Usage}\label{run-command-usage}

\begin{lstlisting}[breaklines]
<create a virtual environment>
$ python3 -m venv spamfilter

<activate the virtual environment>
$ source spamfilter/bin/activate

<install the requirement>
$ python3 -m pip install -r requirements.txt

<run the training code>
$ python3 code/training.py -i <spam.csv file> -os <output spam probability file> -oh <output ham probability file>

<classify the test set>
$ python3 code/classify.py -i <testing dataset .csv file> -is <spam probability file> -ih <ham probability file> -o <classification output filename>
        
<add new training to training set>
$ python3 code/addtotraining.py -is <input spam probability file> -ih <input ham probability file> -s <new training set file>
\end{lstlisting}

Some notes,

\begin{itemize}
    \itemsep1pt\parskip0pt\parsep0pt
    \item Creating and activating virtual environment are optional steps.
    \item The first row of the input file is counted as a header. So every input file should have a header row; otherwise, the first row will be excluded from data.
\end{itemize}

\section*{Examples to Run}\label{examples-to-run}
\begin{lstlisting}[breaklines]
<to run train dataset>
$ python3 code/training.py -i data/spam.csv -os data/spam_probability.csv -oh data/ham_probability.csv

<to classify test data>
$ python3 code/classify.py -i data/test.csv -is data/spam_probability.csv -ih data/ham_probability.csv -o data/output.csv

<to add new dataset to trained data>
$ python3 code/addtotraining.py -is data/spam_probability.csv -ih data/ham_probability.csv -s data/new_training.csv
    
\end{lstlisting}
\end{document}
